%& --translate-file=cp1250pl
\documentclass[11pt,a4paper]{article}
\usepackage[left=3.5cm, right=4cm, bottom=6.65cm, top=5cm]{geometry}

\usepackage{amssymb}
\usepackage{amsmath}
\usepackage{setspace}
\usepackage{setspace}
\usepackage{array}
\usepackage{longtable}
\usepackage{tabularx}
\usepackage{multicol}
\usepackage{graphics}
\usepackage{graphicx}
\usepackage{times}
\usepackage{algorithmic}
\usepackage{algorithm}

%=========================================================================%
%============================== Definitions ==============================%
%=========================================================================%

\makeatletter

\bibliographystyle{plain}
\newtheorem{theorem}{Theorem}
\newtheorem{proof}{Proof}
\newtheorem{lemma}{Lemma}
\newtheorem{property}{Property}
\newtheorem{corollary}{Corollary}
\newtheorem{procedure}{Procedure}


\def\tablename{Table}
\def\figurename{Fig.}

\pagestyle{empty}

\renewcommand{\theenumi}{\roman{enumi})}
%\def\secsize{18pt}

\renewcommand{\thefootnote}{\fnsymbol{footnote}}


\newcommand\sectionsize{\fontsize{11}{12}\selectfont}
\newcommand\subsectionsize{\fontsize{9}{12}\selectfont}
\newcommand\refefencesize{\fontsize{10}{12}\selectfont}

\newcommand{\Booktitle}[1]{
\begin{flushleft}
\fontsize{10pt}{0pt}\selectfont \fontfamily{\sfdefault}\noindent \vspace{1pt}  \textbf{#1} \fontfamily{\familydefault} \\[12pt]
\end{flushleft}}

\newcommand{\Keywords}[1]{
\begin{flushright}
{\fontsize{9pt}{0pt}\selectfont \textit{Keywords: #1}} \\[13pt]
\end{flushright}}

\newcommand{\Title}[1]{
{\fontsize{13pt}{15pt}\selectfont
\begin{center}
 \textbf{#1} \\[36pt]
\end{center}}}

\newcommand{\Abstract}[1]{\setlength{\parindent}{5mm}
\parbox{13cm}{ \fontsize{9pt}{12pt}\selectfont \setlength{\parindent}{5mm} #1}}


%\renewcommand\caption{\fontsize{9}{1}\selectfont #1}

\newcommand{\captionfonts}{\fontsize{9}{1}\selectfont}

\setlength\abovecaptionskip{0pt}
\setlength\belowcaptionskip{0pt}
\newlength\abovetableskip
\newlength\belowtableskip
\newlength\abovefigureskip
\newlength\belowfigureskip
\setlength\abovetableskip{0pt}
%
\setlength\belowtableskip{6pt}
%
\setlength\abovefigureskip{6pt}
%
\setlength\belowfigureskip{0pt}

\long\def\@makecaption#1#2{%
  \vskip\abovecaptionskip
  \sbox\@tempboxa{{\captionfonts #1. #2 }}%
  \ifdim \wd\@tempboxa >\hsize
    {\captionfonts #1. #2\par}
  \else
    \hbox to\hsize{\hfil\box\@tempboxa\hfil}%
  \fi
  \vskip\belowcaptionskip}

\renewenvironment{table}{%
   %\let \@makecaption \@maketablecaption
   \setlength{\abovecaptionskip}{\abovetableskip}
   \setlength{\belowcaptionskip}{\belowtableskip}
   \@float{table}}%
   {\end@float}

\renewenvironment{figure}{%
  \setlength{\abovecaptionskip}{\abovefigureskip}
  \setlength{\belowcaptionskip}{\belowfigureskip} \vspace{0pt}
  \@float{figure}}%
  {\end@float}


\renewcommand\section{\@startsection {section}{1}{0cm}%
                                   {24pt}%
                                   {15pt}%
                                   {\centering\normalfont\sectionsize}}

\renewcommand\subsection{\@startsection {subsection}{2}{0cm}%
                                   {9pt}%
                                   {9pt}%
                                   {\centering\normalfont\subsectionsize}}

\renewcommand{\@seccntformat}[1]{\csname the#1\endcsname.\quad }

\def\refname{{\sectionsize REFERENCES}}
\def\refsize{\refefencesize}


%=========================================================================%
%========================== Author(s) & Title ============================%
%=========================================================================%

\begin{document}
%
\Booktitle{Computer Systems Engineering 2007 (co tu wpisac?)}
%
\Keywords{american sign language, images recognition, machine learning}
%
\noindent Mi\l osz BIA\L CZAK\footnote{\noindent Wroc\l aw University of Science and Technology, Poland} \\
%
\noindent Martyna \L AGO\.ZNA\footnote{\noindent Wroc\l aw University of Science and Technology, Poland} \\[7pt]
%
\Title{AMERICAN SIGN LANGUAGE RECOGNITION}


%=========================================================================%
%============================== Abstract =================================%
%=========================================================================%

\Abstract{
	What if the fast development of computer science, especially machine learning could help disabled people? In fact, it can and this is the topic to which the research described in this paper has been devoted. Deaf-mute people are the part of our society and it would be a great convenience both for them and speaking people which would allow for a better communication using technology.
	
	In this paper, the results of research concerning recognition of sign language has been shared. The research includes experimenting with the images transformations and usage of different learning and features detecting algorithms to obtain the best quality of signs recognition. In addition, the impact of background and different hands rotations on the accuracy of received results has been also the part of this work.
}


%=========================================================================%
%=========================== INTRODUCTION ================================%
%=========================================================================%

\section{INTRODUCTION}

Ludzie nie niemi, mają problem porozumienia się z ludźmi niemymy. Ichcieluśmy to uprościć.

%=========================================================================%
%========================= PROBLEM FORMULATION ===========================%
%=========================================================================%

\section{PROBLEM FORMULATION}

Problemem jest:

- znalezienie cech w obrazku

- uwydatnienie cech ręki

- uczenie algorytmu

%=========================================================================%
%============================= ALGORITHMS ================================%
%=========================================================================%

\section{ALGORITHMS}

\subsection{Preprocessing}

- anisotropic filtering

- gaussian blur

\subsection{Features extraction}

- ORB

- BRIEF

- CENSURE 

\subsection{Postprocessing}

- Normalize

- Standard Scaler 

\subsection{Learning}

SVM

- points

- vector

- combined

%=========================================================================%
%======================== RESULTS OF EXPERIMENTS =========================%
%=========================================================================%

\section{RESULTS OF EXPERIMENTS}

Wykresy dla danych z Politechniki Śląskiej i dla naszych.


%=========================================================================%
%============================== CONCLUSION ===============================%
%=========================================================================%

\section{CONCLUSION}



%=========================================================================%
%============================== CONCLUSION ===============================%
%=========================================================================%

\begin{thebibliography}{99}
\refefencesize \setlength\baselineskip{5pt}
%
\bibitem{CR76} CARLSON J.G. and ROWE R.G., \textit{How much does forgetting cost?} Industrial Engineering, vol. 8, 1976 pp. 40--47.
\bibitem{GG40} GRAHAM C.H. and  GAGNE R.M., \textit{The acquisition, extinction, and spontaneous recovery of conditioned operant response}. Journal of Experimental Psychology, vol. 26, 1940, pp. 251--280.
\bibitem{JB99} JABER Y.M. and BONNEY M., \textit{The economic manufacture/order quantity (EMQ/EOQ) and the learning curve: Past, present, and future}. International Journal of Production Economics, vol. 59, 1999, pp. 93--102.

\end{thebibliography}



\end{document}